\documentclass[CJK]{beamer}
\input{macros.tex}
\begin{document}
\bch

\title{第七课  LAPACK,SVD和最佳线性拟合}
\subtitle{http://zhiqihuang.top/algorithms}
  \author{黄志琦}
  \date{}

  \maketitle

  \begin{frame}
    \frametitle{内容摘要}
  \tableofcontents
  \end{frame}

  \section{LAPACK}

  \begin{frame}
    \frametitle{解决线性代数问题的标准库 - LAPACK}
    在正儿八经的大规模科学计算中,线性代数的问题都用{\blue Linear Algebra PACKage (LAPACK)}这个(牛炸天的)程序库来解决。

    \skiplines
    
    大多数同学还在使用windows系统,安装使用LAPACK很麻烦(好吧我承认我跟微软不大对付)。

    \skiplines
    
    所以我们用一个替代品:{\blue python的linalg库}来讲解线性代数的问题。(实际上linalg用的计算内核还是LAPACK。)
  \end{frame}


  \begin{frame}
    \frametitle{Python中的linalg库能干些啥}
    \bitem
  \item{矩阵和矢量的常规操作}
  \item{解线性方程组,求矩阵的逆}
  \item{解决矩阵的本征值问题}
  \item{{\blue 矩阵的SVD}}
  \item{{\blue 解决最佳线性拟合问题}}    
    \eitem

    前面的几种操作线性代数课里都有详细的论述,在linalg里调用也非常容易(请自行搜索练习)。我们要着重讲解的是矩阵的SVD和最佳线性拟合问题。
  \end{frame}
  

  \section{SVD}


  \begin{frame}
    \frametitle{标准化的准对角矩阵}
    如果一个 $m\times n$矩阵$A$的左上最大子方阵(由$A$的$1$至$\min(m, n)$行和$1$至$\min(m, n)$列的公共部分组成)是对角矩阵,且对角元素是逐行递减的非负实数($A_{1,1}\ge A_{2,2}\ge \ldots\ge 0$),则称$A$是{\blue 标准化的准对角矩阵}。

    \skiplines

    例子:
    
    \bmini{0.25}
    \bcenter
    $3\times 3$,秩为$2$
    \ecenter
   \begin{equation}\left(\begin{array}{lll}
    3 & 0 & 0 \\
    0 & 2 & 0 \\
    0 & 0 & 0 
     \end{array}\right)\nonumber\end{equation}

     \vspace{0.2in}
    \emini    
    \bmini{0.3}
    \bcenter
    $5\times 3$,秩为$3$
    \ecenter
   \begin{equation}\left(\begin{array}{lll}
    3 & 0 & 0 \\
    0 & 1 & 0 \\
    0 & 0 & \frac{1}{2} \\
    0 & 0 & 0 \\
    0 & 0 & 0                         
     \end{array}\right)\nonumber\end{equation}
    \emini
    \bmini{0.38}
    \bcenter
    $4\times 6$,秩为$2$    
    \ecenter
   \begin{equation}\left(\begin{array}{llllll}
     2 & 0 & 0 & 0 & 0 & 0 \\
    0 & 1 & 0 & 0  & 0 & 0 \\
    0 & 0 & 0 & 0  & 0 & 0 \\
    0 & 0 & 0 & 0  & 0 & 0    
     \end{array}\right)\nonumber\end{equation}

     \vspace{0.1in}     
       \emini
    
  \end{frame}


    \begin{frame}
    \frametitle{single value decomposition (SVD)的定义}

    定理:一个$m\times n$的实矩阵$A$,总是能够分解为如下形式:

    $$ A = U \Sigma V^T $$
    其中$U$是$m\times m$的正交矩阵 ($U U^T = I_{m\times m}$),$V$是$n\times n$的正交矩阵 ($VV^T = I_{n\times n}$)。$\Sigma$是$m\times n$标准化的准对角矩阵。

    (证明略)

    \skipline
    
    这个分解就是名满江湖的single value decomposition ({\blue SVD})。如果要数出5个线性代数最实用理论,SVD一定是其中一个。
    \end{frame}


    \begin{frame}
    \frametitle{SVD的几何含义}
    设$A$是一个$m\times n$的矩阵。映射$\vecx\rightarrow A\vecx$把一个$n$维的“初始空间”里的向量$\vecx$映射到一个$m$维的“目标空间”里的$A\vecx$。

    \skipline
    
    下面我们来通过SVD理解这种不同维空间的映射到底是如何完成的。

    \skipline
    
    设$A$的SVD为:
    $$ A = U \Sigma V^T $$
    $V$的列矢量$\vecv_1, \vecv_2,\ldots, \vecv_n$构成初始空间的一组标准正交基。
 
    $U$的列矢量$\vecu_1, \vecu_2,\ldots, \vecu_n$构成目标空间的一组标准正交基。

    $\Sigma$的准对角元$\sigma_1\ge\sigma_2\ge\ldots\ge\sigma_{\min(m,n)}\ge 0$可以看成一些权重,权重的个数不超过$\min(m,n)$个。非零权重的个数(记为$r$)为$\Sigma$的秩,实际上也是$A$的秩。
    \end{frame}

    \begin{frame}
    \frametitle{SVD的几何含义}    
    如果初始空间的矢量$\vecx$可以按标准正交基分解为
    $$\vecx = VV^T\vecx = V\vecy = y_1 \vecv_1  + y_2 \vecv_2 + \ldots + y_n\vecv_n.$$
    那么
    $$ A \vecx = U\Sigma V^T\vecx = U \Sigma \vecy = (\sigma_1y_1)\vecu_1 + (\sigma_2y_2)\vecu_2+\ldots $$
    也就是说,如果用$V$的列向量为正交基的坐标系描述初始空间,用$U$的列向量为正交基的坐标系描述目标空间,那么映射$A$只是把每个坐标分量依次乘了权重$\sigma_1$, $\sigma_2$,\ldots而已。

    \skipline

    因为在初始空间中取$V$列向量为基的坐标系和在目标空间中取$U$列向量为基的坐标系非常地合理和自然,我们不妨称这些坐标系为“合理坐标系”。


    \end{frame}


    \section{Linear Least Square}
    
    \begin{frame}
      \frametitle{SVD和最佳线性拟合的问题}
      我们经常关心最佳线性拟合问题:对给定的映射$A$以及目标空间里的矢量$\vecb$,如何找到一个初始空间的矢量$\vecx$,使$A\vecx$尽可能地接近$\vecb$。也就是要使$\lVert A\vecx - \vecb\rVert$最小。

      \skipline
      
      有了上述几何观点之后,这个问题现在看起来很容易,不是吗?对$b$的每个“合理坐标系坐标”$b_i$,如果对应的权重$\sigma_i$不为零,则只要取$\vecx$的相应合理坐标系坐标为$b_i/\sigma_i$。对那些权重为零的分量,就可以都取零(当然也可以随便取,不过一般来说我们希望$\vecx$不要包含冗余信息,所以还是取零吧)。如果$\vecb$有维度超出初始空间的分量,就根本不用管了。
      
    \end{frame}


    
    
    



    \begin{frame}
      \frametitle{最佳线性拟合的问题}
      所以求解最佳线性拟合的步骤为:
      \bitem
      \item{SVD: $$A = U\Sigma V^T$$ }
      \item{求$b$的“合理坐标”: $$ \widetilde{\vecb} = U^Tb $$}
      \item{计算$x$的“合理坐标”: $\widetilde{x}_i = \widetilde{b}_i/\sigma_i$ (如果$\sigma_i \ne 0$)或者$\widetilde{x}_i=0$ (如果$\sigma_i=0$).}
      \item{计算$x$的原始坐标系坐标: $\vecx  = V \widetilde{\vecx}$.}
        \eitem

        \skipline
        
        {\bf 注意:}在数值计算时,因为有浮点误差,碰到很小但非零的$\sigma_i$就需要小心:它很可能只是浮点误差产生的,不具有物理意义(具体要根据实际模型来判断),这时需要把它当成零来处理。
    \end{frame}


    \begin{frame}
      \frametitle{linalg的最佳线性拟合函数}
      python的linalg.lstsq集成了上述最佳线性拟合的所有步骤,请自行搜索学习这个函数的用法。在lstsq里当然也允许你手动设置当$\sigma_i$很小时(小于参数rcond)就把它忽略。

      \skipline

      linalg库里也有单独只做SVD的函数linalg.svd。请自行搜索学习用法。
      
    \end{frame}


    \begin{frame}
      \frametitle{补充说明}
      既然python的linalg.lstsq集成了最佳线性拟合的所有步骤,为什么还要唠叨一大堆介绍SVD呢?

      \skiplines
      
      这是因为SVD的几何诠释清晰地告诉你了线性映射$A$做的事情。在实际应用中,很多时候需要先理解模型,再进行数值计算。所以SVD这个中间步骤非常有意义。
      
    \end{frame}

    \section{Applications}
    
    \begin{frame}
      \frametitle{最佳线性拟合的应用}
      家喻户晓的“最小二乘法”的本质是用两个函数
      $$f_1(x) = 1$$
      和
      $$f_2(x) = x$$
      的线性组合来拟合一个和数据符合得最佳的函数 $f(x) $。
    \end{frame}


    \begin{frame}
      \frametitle{最佳线性拟合的应用}
      一般地,可以用任意一组线性独立的函数
      $$f_1(x), \ldots, f_{n}(x).$$
      的线性组合来拟合一个和数据符合得最佳的函数 $f(x)$。


      设数据点有$m$个
      $$(x_1, y_1), (x_2, y_2), \ldots, (x_m, y_m).$$
      设最佳拟合的函数为
      $$ f(x) = c_1 f_1(x) + c_2f_2(x) + \ldots + c_n f_n(x).$$
      用前面学习的套路可以找到$\mathbf{c} = (c_1, c_2,\ldots, c_n)^T$ 令 $\sum_i [y_i - f(x_i)]^2$,也就是$\lVert A\mathbf{c} - \vecy\rVert$最小化。这里$m\times n$的$A$定义为:
      $$A_{i,j} = f_j(x_i)$$
      
    \end{frame}


    \begin{frame}
      \frametitle{作业}
      在$[-\frac{\pi}{2},\frac{\pi}{2}]$上取均匀地取$m=101$个点:
      $$x_0=-\frac{\pi}{2},\ x_1 = -\frac{\pi}{2}+\frac{\pi}{100},\ x_2 = -\frac{\pi}{2}+\frac{2\pi}{100}, \ldots, x_{100} = \frac{\pi}{2}.$$

      制造模拟数据点:
      $$ y_i= \sin(x_i), \ (i=0,1,2,\ldots, 100).$$

      用四次多项式
      $$ y = c_0 + c_1x+c_2x^2+c_3x^3+c_4x^4 $$
      来对这些数据点做最佳线性拟合。把求出的$c_0,c_1,c_2,c_3,c_4$和$\sin x$的泰勒展开系数做比较。
      
    \end{frame}
  \ech
\end{document}
