\documentclass[CJK]{beamer}
\input{macros.tex}
\begin{document}
\bch

\title{第六课  高维空间近邻点搜索}
\subtitle{http://zhiqihuang.top/algorithms}
  \author{黄志琦}
  \date{}

  \maketitle

  \begin{frame}
    \frametitle{内容摘要}
  \tableofcontents
  \end{frame}

  \section{Nearest Neighbors Search}

  \begin{frame}
    \frametitle{问题}
    在$k$维欧氏空间有一些给定的点:$\vecx_1$, $\vecx_2$, \ldots, $\vecx_N$。对任意的一个测试点$\vecy$要求寻找离$\vecy$最近的$m$个点. $m\ll N$是个给定的很小的整数(通常不到10)。

    \skipline
    
    因为要对很多测试点寻找$m$最近邻,希望算法尽可能地优化单次搜寻时间。
  \end{frame}


  \begin{frame}
    \frametitle{笨办法}
    把所有点按照和测试点的距离从小到大排序,然后取前$m$个。

    \skiplines

    准备工作复杂度:不需要准备工作。
    
    单次搜索复杂度:$O(N\log N)$
  \end{frame}
  

  \section{$k$-d Tree}
  
  \begin{frame}
    \frametitle{$k$-d Tree的基本思想}

    在各个维度上同时进行二分搜索,以期达到$O(\log N)$的复杂度。
    
  \end{frame}


  \begin{frame}
    \frametitle{$k$-d Tree的构造方法}

    依次在各个维度的坐标寻找中位数,把含中位数的一个点放到当前节点,把小于中位数的放到当前节点的左分枝,把其他点放到当前节点的右分枝。


    \skiplines
    
    准备工作:$O(N^2)$ (如果牺牲一些搜索速度,还可以用近似方法减少这部分的复杂度。)
    
  \end{frame}


  \begin{frame}
    \frametitle{$k$-d Tree的搜索$m$个最近邻的方法}
    \bitem
  \item{从树根出发,每移动一次都检验更新$m$个最近邻。}
    \item{先往测试点的应选(就是如果小于当前节点对应的划分维度坐标,就选左;否则选右)分枝移动搜索(调用函数自身递归)。}
    \item{再通过比较{\blue 测试点和节点所在的划分超平面的距离}和当前$m$最近邻距离的最大的一个,确定是否要搜索另一个分枝。如果需要,则进入另一个分枝进行搜索(调用自身递归)。}
      \eitem


      \skiplines
      
      单次搜索平均复杂度: $O(\log N)$
    
  \end{frame}


  \begin{frame}
    \frametitle{作业}

    在所给的范例基础上,写两个子函数分别实现$k$-d tree增加一个元素的操作和删除一个元素的操作。
    
  \end{frame}
  

  \section{Remarks}
  
  \begin{frame}
    \frametitle{补充说明}

    \bitem
  \item{当删除或者增加元素操作过多时,$k$-d tree会变得“不平衡”(即某些左枝的深度远远大于右枝的深度)。这时需要重新平衡一下$k$-d tree。有兴趣的同学可以搜索一下这方面的算法。}
  \item{当维度$k$非常大(至少大于20)时,$k$-d tree的搜索效率变低,这时可以使用一种叫ball tree的数据结构。详细算法请自行搜索了解。}
    \eitem

    
  \end{frame}
  
  
  
  \ech
\end{document}
