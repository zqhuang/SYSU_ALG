\documentclass[CJK,14pt]{beamer}
\input{macros.tex}
\begin{document}
\bch

\title{第六课  近邻点搜索和$k$-d Tree}
\subtitle{http://zhiqihuang.top/algorithms}
  \author{黄志琦}
  \date{}

  \maketitle

  \begin{frame}
    \frametitle{内容摘要}
  \tableofcontents
  \end{frame}

  \section{近邻点搜索}

  \begin{frame}
    \frametitle{近邻点搜索的应用场景}
    考虑在N维空间做插值计算。如果数据点是无规律分布的,一般会考虑找最近的若干个数据点进行加权迭加。    
  \end{frame}

  \begin{frame}
    \frametitle{近邻点搜索的应用场景}

    
  \end{frame}

  
  \ech
\end{document}
