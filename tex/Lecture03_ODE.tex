\documentclass[CJK,14pt]{beamer}
\input{macros.tex}
\begin{document}
\bch

\title{第三课 常微分方程}
\subtitle{http://zhiqihuang.top/algorithms}
  \author{黄志琦}
  \date{}

  \maketitle

  \begin{frame}
    \frametitle{内容摘要}
  \tableofcontents
  \end{frame}


  \section{Generic Form}
  
  \begin{frame}
    \frametitle{常微分方程(Ordinary Differential Equations)的一般形式}
    常微分方程都能写成:
    $$\frac{d\vecx}{dt} = f(t, \vecx).$$
    的形式。
  \end{frame}


  \begin{frame}
    \frametitle{例子}
    简谐振动:
    $$ \frac{d^2y}{dt^2} + \omega^2 y = 0.$$
    定义矢量$(x_1, x_2) = (y, \frac{dy}{dt})$,则
    \bea
    \frac{d x_1}{dt} &=& x_2 \newl
    \frac{d x_2}{dt} &=& - \omega^2 x_1 .    
    \eea
  \end{frame}
  

  \section{4th order Runge-Kutta}

  \begin{frame}
    \frametitle{4阶Runge-Kutta算法}
    设时间步长为$h$,
    \bea
    \veck_1 &=& f(\vecx, t), \newl
    \veck_2 &=& f(\vecx + \frac{h}{2}\veck_1, t+\frac{h}{2}), \newl
    \veck_3 &=& f(\vecx + \frac{h}{2}\veck_2, t+\frac{h}{2}),  \newl
    \veck_4 &=& f(\vecx + h \veck_3, t+h),  \newl
    \vecx &=& \vecx + \frac{h}{6}\left(\veck_1+2(\veck_2+\veck_3)+\veck_4\right)
    \eea
    单步误差$O(h^4)$
  \end{frame}

  
  \section{The Mysterious DVERK }

  \begin{frame}
    \frametitle{神秘的DVERK程序}
  DVERK由几个算法大牛写于60年代,古老的fortran66语言,goto语句满天飞,完全看不懂;但仍是目前综合性能最佳的解ODE神器。    
    \bitem
  \item{5阶/6阶混合变步长Runge-Kutta算法。}
  \item{精度可调(内置误差估算)。}
  \item{如果导数处处连续,DVERK稳定,快速,准确可靠。}
  \item{如果导数不连续,DVERK不太稳定,需要对导数作一些近似的预光滑处理。}
    \eitem

  \end{frame}


  \begin{frame}
    \frametitle{范例程序}
    \url{zhiqihuang.top/algorithms/examples.php}
  \end{frame}
  
  \ech
\end{document}
