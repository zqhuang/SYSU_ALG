\documentclass[CJK,14pt]{beamer}
\input{macros.tex}
\begin{document}
\bch

\title{第一课:排序算法}
\subtitle{http://zhiqihuang.top}
  \author{黄志琦}
  \date{}

  \maketitle

  \begin{frame}
    \frametitle{内容摘要}
  \tableofcontents
  \end{frame}

  \section{综述 (Introduction)}

  \begin{frame}
\frametitle{为什么要学习(令人头疼的)算法?}


    \addfig{1}{tuding.jpg}
    
  \end{frame}

  
  \begin{frame}
    \bitem
  \item{观点A: 学习算法还不如学习一门新的编程语言。}
    \eitem

    \addfig{3}{cv.jpg}
  \end{frame}
  
  
  \begin{frame}
    \bitem
  \item{观点B: 常见的算法都有现成封装好的库,直接调用就行了。}
    \eitem

{\color{violet}
    $\gg$ x = [ 4, 2, 1, 10, 5 ]
    
    $\gg$ x.sort()
    
    $\gg$  print x

  [1, 2, 4, 5, 10]
}

    
  \end{frame}


  \begin{frame}
    你有幸成为了阿里公司的码农,从此走向人生巅峰……

    \addfig{2}{manong.jpg}
  \end{frame}

  \begin{frame}
\frametitle{老板给你的任务}
    淘宝服务器上有很多地区电话号码,每个地区的电话号码(假设都是7位正整数)都无序地存在一个(对应于该地区的)文件里。现在要求你写一个程序,把任给的一个文件中的电话号码读出来,排序,然后重新写入文件。

    因为服务器要负担很多其他的工作,老板要求你的程序运行时间不超过1秒,占用内存少于2MB。你能完成任务吗?
    
  \end{frame}


  \begin{frame}
\frametitle{C和C++的解决方案}


http://zhiqihuang.top/code/sort/bitmap.c

\skiplines

http://zhiqihuang.top/code/sort/bitmap.cpp

  \end{frame}
  
  \begin{frame}
    \frametitle{学习算法的重要性}
    \bitem
  \item{算法是逻辑思考能力,是写任何程序的前提;编程语言一通则百通。}
  \item{现成的库通用性好,对特定的问题往往速度较慢,有时候甚至未必能够解决问题。}
    \eitem
  \end{frame}
  
  
  \section{冒泡排序 (Bubble Sort)}

  \begin{frame}
    \frametitle{}
  \end{frame}

  \section{快速排序 (Quick Sort)}

  \begin{frame}
    \frametitle{}
  \end{frame}

  \section{归并排序 (Merge Sort)}

  \begin{frame}
    \frametitle{}
  \end{frame}
  
\ech
\end{document}
