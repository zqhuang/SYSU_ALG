\documentclass[CJK,14pt]{beamer}
\input{macros.tex}
\begin{document}
\bch

\title{第四课:最短路径问题}
\subtitle{http://zhiqihuang.top/algorithms}
  \author{黄志琦}
  \date{}

  \maketitle

  \begin{frame}
    \frametitle{内容摘要}
  \tableofcontents
  \end{frame}

  \section{Min-distance problems}

  \begin{frame}
    \frametitle{寒假快到了,苗十三准备去旅游了!}


    因为这学期被{\bf 扣工资扣得很惨},苗十三手头不是很宽裕,必须精打细算。他想弄清去各个城市的最低开销。

    {\scriptsize(嗯,不用考虑回来的开销。苗十三准备找警察叔叔说自己被拐卖,免费被送回来。)}


  \end{frame}


  \begin{frame}
    \frametitle{苗十三查到的各个航班的机票价格}
    \addfig{4}{paths.jpg}
  \end{frame}
  

  \begin{frame}
    \frametitle{用矩阵来表示}
    用0,1,2,\ldots,7分别表示珠海,上海,北京,广州,重庆,南京,杭州,拉萨。

    \skiplines
    
    用一个二维数组 x[8][8] 来表示航班价格:

    \skiplines
    
    x[i][j] = i到j的直飞价格(如无航班记作$\infty$)

  \end{frame}

  \section{Floyd-Warshall Algorithm}

  \begin{frame}
    \frametitle{Floyd-Warshall算法思想}

    第一步:把矩阵更新为

    x[i][j] = 允许到城市0中转,i到j的最少机票花销

    {\blue \small
  if(x[i][0]+x[0][j]$<$x[i][j]) x[i][j]=x[i][0]+x[0][j];}

\skipline

第二步:把矩阵更新为

x[i][j] = 允许到城市0和1中转,i到j的最少机票花销

{\blue \small
  if(x[i][1]+x[1][j]$<$x[i][j]) x[i][j]=x[i][1]+x[1][j];}
    
{\scriptsize 注意:如果最佳转机顺序是i$\rightarrow$1$\rightarrow$0$\rightarrow$j.因为x[1][j]已经在第一步被更新为 i$\rightarrow$0$\rightarrow$j的价格,在第二步仍然可以通过比较x[i][1]+x[1][j]和x[i][j]找到正确的最低价格。}

这样一直进行下去,第八步之后(搞定):

x[i][j] = 允许任意转机, i到j的的最少机票花销
  \end{frame}
  
  \begin{frame}
    \frametitle{Floyd-Warshall算法的优缺点}
    \bitem
  \item{优点:算法及其简明;可以求出图中任意两点最短距离。}
  \item{缺点:对$N$个顶点的图,算法复杂度是$O(N^3)$,不适合太多个顶点的问题。}
    \eitem
  \end{frame}

  \begin{frame}
    \frametitle{程序实现}

    参考:
    
    zhiqihuang.top/algorithms/examples.php
    
    
  \end{frame}

  \section{Dijkstra Algorithm}
  
  \begin{frame}
    Floyd-Warshall算法一下子把任何两个城市间的最低机票开销算出来了。但苗十三只关心从珠海到各个城市怎么飞最便宜。有没有不那么“浪费”的算法呢?

    \skiplines
    
    有的,比如Dijkstra算法就是个不错的方法。
  \end{frame}
  

  \begin{frame}
    \frametitle{Dijkstra算法的基本思想}
    用一个额外的数组m来记录珠海到各个城市的最少机票开销:

    如果要求直飞:

    m=\{$600,900,200,1100,\infty,\infty,\infty$\};

      
    
  \end{frame}


  \begin{frame}
    \frametitle{Dijkstra算法的基本思想}
    注意到珠海到广州的直飞价格200最低。显然,即使允许任意转机,这个价格也是最低的(因为一往其他地方飞,就超过这个价格了)。我们用蓝色标记出来,表示这个价格已经是允许任意转机的最低价了:

    m=\{$600,900,{\blue 200},1100,\infty,\infty,\infty$\};      

    接下来,我们立刻考虑能否通过从广州转一次机省钱:从广州可以直接到达的只有北京和拉萨,经过比较发现可以把到北京和拉萨的最低开销都更新:

    m = \{ $600, 800, {\blue 200}, 1100, \infty, \infty, 1700$ \};

      
    
  \end{frame}


  \begin{frame}
    \frametitle{Dijkstra算法的基本思想}
    现在还没有标记蓝色的数字中最小的是珠海到上海的价格600。这个其实也已经是允许任意转机的最低价了!为什么呢?因为要比这个价格低,只可能从价格更低的广州转机。而从广州转机的价格我们已经考虑并更新过了。

    OK,既然是允许任意转机的最低价,我们继续标记蓝色:
    

    m = \{ ${\blue 600}, 800, {\blue 200}, 1100, \infty, \infty, 1700$ \};

    可以从上海转机去重庆和南京,我们立刻比较更新一下去重庆和南京的最低开销:

    m = \{ ${\blue 600}, 800, {\blue 200}, 1000, 800, \infty, 1700$ \};    
  \end{frame}


\begin{frame}
  \frametitle{Dijkstra算法的基本思想}
  现在的
  
  m = \{ ${\blue 600}, 800, {\blue 200}, 1000, 800, \infty, 1700$ \};

  记录的是允许从广州和上海转机的最低开销。还没蓝染(em?)的城市中最低开销有北京800和南京800。这两个也都已经是允许任意转机的最低开销了!都蓝染一下:
  
  m = \{ ${\blue 600}, {\blue 800}, {\blue 200}, 1000, {\blue 800}, \infty, 1700$ \};  

  可以从北京和南京转机去上海(已经蓝了别指望更新了),杭州,拉萨;比较后发现可以把去杭州最低开销更新一下:

    m = \{ ${\blue 600}, {\blue 800}, {\blue 200}, 1000, {\blue 800}, 1100, 1700$ \};  
  \end{frame}
    

\begin{frame}
  \frametitle{Dijkstra算法的基本思想}
  继续找黑色的最小那个数,重庆:1000。标上蓝色:

  m = \{ ${\blue 600}, {\blue 800}, {\blue 200}, {\blue 1000}, {\blue 800}, 1100, 1700$ \};

  可以从重庆转机去的南京已经是蓝色,不需要考虑更新。
\end{frame}


\begin{frame}
  \frametitle{Dijkstra算法的基本思想}
  继续找黑色的最小那个数,杭州:1100。标上蓝色:

  m = \{ ${\blue 600}, {\blue 800}, {\blue 200}, {\blue 1000}, {\blue 800}, {\blue 1100}, 1700$ \};

  可以从杭州转机去的上海已蓝,跳过。
\end{frame}


\begin{frame}
  \frametitle{Dijkstra算法的基本思想}
  最后黑色的拉萨:1700.标上蓝色:

  m = \{ ${\blue 600}, {\blue 800}, {\blue 200}, {\blue 1000}, {\blue 800}, {\blue 1100}, {\blue 1700}$ \};

  最后的答案就是这些全部染蓝的数字!
\end{frame}


\begin{frame}
  \frametitle{Dijkstra算法的复杂度}
  假如每个顶点出去的边数不是很多的话,因为每次可以用二分法搜索“黑色最小数”,复杂度就是$O(N\log N)$。

  假如每个顶点出去的边数非常多,在最坏的情况下,就会是$O(N^2\log N)$的复杂度;当然,实际应用中这样的情形极少出现。

\end{frame}

\begin{frame}
  \frametitle{作业}
  用Dijkstra算法写程序解决苗十三的旅游问题。
\end{frame}


\section{Remarks}

\begin{frame}
  \frametitle{补充说明}
  我们之前讨论的问题都是边的权重为正的情形。如果存在负权重边(就是航空公司倒贴苗十三请求苗十三飞),就要用一种叫Bellman-Ford的算法。因为应用场景不多,就不介绍了。
\end{frame}

  \ech
\end{document}
