\documentclass[CJK,14pt]{beamer}
\input{macros.tex}
\begin{document}
\bch

\title{第二课:插值和搜索}
\subtitle{http://zhiqihuang.top/algorithms}
  \author{黄志琦}
  \date{}

  \maketitle

  \begin{frame}
    \frametitle{内容摘要}
  \tableofcontents
  \end{frame}

  \section{Interpolation}

  \begin{frame}
    \frametitle{内插值(Interpolation)问题}
               {\bf 模型}: $y$是$x$的光滑函数
               
               {\bf 数据}: 一堆(无误差的)样本$(x_1,y_1)$, $(x_2, y_2)$, \ldots, $(x_N, y_N)$。设其中最小的$x$值为$x_{\min}$,最大的$x$值为$x_{\max}$.

               {\bf 要求}: 写一个函数,要求
               \bitem
               \item{每次调用复杂度不超过$O\left(\log(N)\right)$}
               
               \item{函数输入:任意满足$x_{\min}\le x\le x_{\max}$的一个$x$}
               
               \item{函数输出:函数值$y(x)$}
                 \eitem

注意插值问题和拟合问题有所不同: 拟合问题里样本是允许有误差的,整个拟合函数的参数个数一般远远小于$N$;插值问题的样本是无误差的(所以要求插值函数对输入的样本$x_i$,严格返回样本$y_i$),插值函数的参数个数一般是$O(N)$的量级。
  \end{frame}


  \begin{frame}
    \frametitle{内插值(Interpolation)问题}
    算法思路:
    \bitem
  \item[1]{把数据根据$x_1,x_2,\ldots, x_N$的大小进行排序。}
  \item[2]{对输入的$x$寻找离它最近的几个(一般2个就够了)数据点。}    
  \item[3]{按照一定的光滑性假设在局域对几个数据点拟合一个简单的函数(这个拟合工作也可以是在第2步之前提前做好保存了的)。}
  \item[4]{按照局域几个数据点拟合的函数计算$y(x)$}
    \eitem
  \end{frame}


  \begin{frame}
    \frametitle{常见的插值计算方法}
    \bitem
  \item{Linear Interpolation (线性插值)}
  \item{Cubic Spline Interpolation (三次样条插值)}
    \eitem
  \end{frame}


  \begin{frame}
    \frametitle{Linear Interpolation 的优缺点}
    \bitem
  \item{优点:稳(对光滑性假设不强,几乎处处适用)}
  \item{缺点:精度一般(对很光滑的函数而言,精度不够理想)}
    \eitem
  \end{frame}


  \begin{frame}
    \frametitle{Cubic Spline 的优缺点}
    \bitem
  \item{优点:对光滑函数而言精度很高}
  \item{缺点:不够稳(对不大光滑的函数不太友好)}
    \eitem
  \end{frame}

    \begin{frame}
    \frametitle{例题}
    从文本文件里读入样本(x,y)(每行一组,未排序),写内插值函数

    \url{http://zhiqihuang.top/algorithms/examples.php}
  \end{frame}

  
  \section{DFS and BFS}

  \begin{frame}
    \frametitle{图的深度优先搜索和广度优先搜索}
    \bitem
  \item{深度优先搜索:通常通过递归来完成。先走完一个分支到头,然后倒回来,一旦发现可以走另一条新的分支,就进入新的分支走到头,再往回倒,依此反复,直至走完所有可能路线。}
  \item{广度优先搜索:按层数推进,第零层是所有可能的出发点,第一层是所有第零层往下走一步形成的路线集合,依此反复。这个算法经常被用来求用最少层数到达给定终点的问题。}
    \eitem
  \end{frame}

  \begin{frame}
    \frametitle{例题}
    对给定的一个集合,找出所有元素的所有可能的全排列

    \url{http://zhiqihuang.top/algorithms/examples.php}    
  \end{frame}

  \section{Hash Table}
  
  \begin{frame}
    \frametitle{散列表(哈希表, hash table)的主要想法}
      数组的寻址是$O(1)$的复杂度,远远快于二分法搜索$O(\log(N))$复杂度。散列表的想法就是把一堆不是整数的对象映射到整数(这个映射叫哈希函数,hash function),然后直接数组寻址读取。

        \skipline
        
      但是不管这个映射方法设计得多巧妙,总会出现两个对象映射到同一个整数的情形,这时候就要在一个位置放多个(哈希函数不是太差劲的话,应该不会是太多个)对象,这可以通过链表的形式来解决。
  \end{frame}

  \begin{frame}
    \frametitle{例题}
    给超市的各种东西的价格做一个哈希表。

    \url{http://zhiqihuang.top/algorithms/examples.php}    
  \end{frame}
  
\ech
\end{document}
